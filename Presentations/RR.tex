%%%%%%%%%%%%%%%%%%%%%%%%%%%%%%%%%%%%%%%%%
% Beamer Presentation
% LaTeX Template

\documentclass{beamer}


\mode<presentation> {

% Theme
\usetheme{metropolis}

%\setbeamertemplate{footline} % To remove the footer line in all slides uncomment this line
%\setbeamertemplate{footline}[page number] % To replace the footer line in all slides with a simple slide count uncomment this line

%\setbeamertemplate{navigation symbols}{} % To remove the navigation symbols from the bottom of all slides uncomment this line
}

\usepackage{graphicx} % Allows including images
\usepackage{booktabs} % Allows the use of \toprule, \midrule and \bottomrule in tables
%\usepackage{cite}
\usepackage[numbers]{natbib}
\usepackage{multirow}
\usepackage{hyperref}



%----------------------------------------------------------------------------------------
%	TITLE PAGE
%----------------------------------------------------------------------------------------

\title[Short title]{Reproducible Research} % The short title appears at the bottom of every slide, the full title is only on the title page

\author{Marco Chiapello} % Your name
\institute[Center for Proteomics] % Your institution as it will appear on the bottom of every slide, may be shorthand to save space
{
Center for Proteomics\\
University of Cambridge \\ % Your institution for the title page
\medskip
\textit{mc983@cam.ac.uk} % Your email address
}
\date{\today} % Date, can be changed to a custom date

\begin{document}

\begin{frame}
\titlepage % Print the title page as the first slide
\end{frame}

\begin{frame}
\frametitle{Overview} % Table of contents slide, comment this block out to remove it
\tableofcontents % Throughout your presentation, if you choose to use \section{} and \subsection{} commands, these will automatically be printed on this slide as an overview of your presentation
\end{frame}

%----------------------------------------------------------------------------------------
%	PRESENTATION SLIDES
%----------------------------------------------------------------------------------------

%------------------------------------------------
\begin{frame}
\section{Introduction} % Sections can be created in order to organize your presentation into discrete blocks, all sections and subsections are automatically printed in the table of contents as an overview of the talk
\vspace{50px}
\begin{flushright}
\scriptsize {\bf Replication} is the ultimate standard by which scientific claims are judged \citep{Peng:2011et}\\
\scriptsize The fact that an analysis is reproducible does not guarantee the quality, correctness, or validity of the published results.
\end{flushright}
\end{frame}
%------------------------------------------------

\begin{frame}
\frametitle{What reproducible research is}
\begin{figure}
\includegraphics[scale=0.45]{figures/001.jpg}
\end{figure}
\end{frame}

%------------------------------------------------

\begin{frame}
\frametitle{What reproducible research is}
\begin{figure}
\includegraphics[scale=0.45]{figures/thenamiracleoccurs.jpg}
\end{figure}
\footnotesize
\begin{itemize}
\item This is exactly how it seems when you try to figure out how authors got from a large and complex data set to a dense paper with lots of busy figures. \\Without access to the data and the analysis code, a miracle occurred.
\item And  there should be {\sc no miracles in science.} \citep{Markowetz:2016cs}
\end{itemize}
\end{frame}

%------------------------------------------------

\begin{frame}
\frametitle{What reproducible research is}
\Large\centering $DATA +  ANALYSIS \rightarrow RESULTS$\\
\rule{\textwidth}{0.05pt}\vspace{20px}

\centering{\sc\Large Reproducible vs Replicable}
\begin{table}[]
\centering
\begin{tabular}{cccc}
                                           &                                & \multicolumn{2}{c}{DATA}                          \\ \cline{3-4}
                                           & \multicolumn{1}{c|}{}          & Same         & \multicolumn{1}{c|}{Different}     \\ \cline{2-4}
\multicolumn{1}{c|}{\multirow{2}{*}{CODE}} & \multicolumn{1}{c|}{Same}      & Reproducible & \multicolumn{1}{c|}{Replicable}    \\
\multicolumn{1}{c|}{}                      & \multicolumn{1}{c|}{Different} & Robust       & \multicolumn{1}{c|}{Generalisable} \\ \cline{2-4}
\end{tabular}
\end{table}
\vspace{3px}
\tiny Ref: {\url{https://github.com/KirstieJane/ReproducibleResearch}}
\end{frame}


%------------------------------------------------
\begin{frame}
\frametitle{What reproducible research is}

\Large{Reproducibility/reproduce}\\
\footnotesize A study is reproducible if there is a specific set of computational functions/analyses (usually specified in terms of code) that exactly reproduce all of the numbers in a published paper from raw data.

\Large{Replication/replicate}\\
\footnotesize A study is only replicable if you perform the exact same experiment (at least) twice, collect data in the same way both times, perform the same data analysis, and arrive at the same conclusions.
\rule{\textwidth}{0.05pt}
{\bf Replicability} requires new samples and new data\footnote{\tiny{in particular biological replicates}}, which introduces new variability, and additional risks of errors. {\bf Reproducibility} is, to some extent, a technical challenge, while replication gives the results scientific validity.
\rule{\textwidth}{0.05pt}\\
\vspace{3px}
\tiny Ref: {\url{https://github.com/lgatto/TeachingMaterial/tree/master/_open-rr-bioinfo-best-practice}}
\end{frame}
%------------------------------------------------

\begin{frame}
\section{Reproducible research Reasons}
\vspace{50px}
\begin{flushright}
\scriptsize How does working reproducibly help to achieve more as a scientist \citep{Markowetz:2016cs}\\
\end{flushright}
\end{frame}
%------------------------------------------------

\begin{frame}
\frametitle{Reasons to work reproducibly}
\Large REPRODUCIBILITY\\
Idealist:
\begin{enumerate}
\tiny
\item It is the foundation of science!
\item The world would be a better place if everyone worked transparently and reproducibly!
\end{enumerate}
Realist:
\begin{enumerate}
\footnotesize
\item It helps to avoid disaster
\begin{itemize}
\tiny
\item You need to record in detail how you got there
\item Work reproducibly early on will save you time later
\end{itemize}
\item It makes it easier to write papers
\begin{itemize}
\tiny
\item To have very transparent data and code, it costs just few minutes to spot a mistake (if any)
\end{itemize}
\item It helps reviewers see it your way
\begin{itemize}
\tiny
\item Made the data and well-documented code easily accessible to the reviewers
\end{itemize}
\item It enables continuity of your work
\begin{itemize}
\tiny
\item How can you ensure the continuity of work in your lab if pro- gress is not documented reproducibly?
\item No proof of reproducibility, no result!
\end{itemize}
\item It helps to build your reputation
\begin{itemize}
\tiny
\item To build a reputation for being an honest and careful researcher
\end{itemize}

\end{enumerate}

\end{frame}

%------------------------------------------------
\section{Reproducible research Tools}
%------------------------------------------------

\begin{frame}
\frametitle{Tools}
Literate programming is a methodology that combines a programming language with a documentation language, thereby making programs more robust, more portable, more easily maintained, and arguably more fun to write than programs that are written only in a high-level language.
\end{frame}

%------------------------------------------------

\begin{frame}
\frametitle{Tools}
At the lowest level, working reproducibly just means avoiding beginners? mistakes. Keep your project organized, name your files and directories in some informative way, store your data and code at a single backed-up location. Don?t spread your data over different servers, laptops and hard drives.
To achieve the next levels of reproducibility, you need to learn some tools of computational reproducibility [8]. In general, reproducibility is improved when there is less clicking and pasting and more scripting and coding. For example, do your analysis in R (https://www.r-project.org/) or Python (https:// www.python.org/) and document your analysis using knitR (http://yihui.name/knitr/) or IPython notebooks (http:// ipython.org/). These tools help you to merge descriptive text with analysis code into dynamic documents that can be automatically updated every time the data or code change.
As a next step, learn how to use a version-control system like git (https://git-scm.com/) on a collaborative platform such as GitHub (https://github.com/). Finally, if you want to become a pro, learn to use docker (http://www.docker.com/), which will make your analysis self-contained and easily transportable to different systems.
\end{frame}


%------------------------------------------------
\begin{frame}
\frametitle{Tools}


Learning the tools of the trade will require commitment and a massive investment of your time and energy. A priori it is not clear why the benefits of working reproducibly outweigh its costs.

Does reproducibility sound like extra work? It can be, particularly when one is first trying to do it, that is, to break one's own previous nonreproducible habits


\end{frame}

%------------------------------------------------
\section{Reproducible research Rules}
%------------------------------------------------
\begin{frame}
\frametitle{Rule 1}


\end{frame}
%------------------------------------------------
\begin{frame}
\frametitle{Rule 2}


\end{frame}
%------------------------------------------------
\begin{frame}
\frametitle{Rule 3}


\end{frame}
%------------------------------------------------
\begin{frame}
\frametitle{Rule 4}


\end{frame}
%------------------------------------------------
\begin{frame}
\frametitle{Rule 5}


\end{frame}
%------------------------------------------------
\begin{frame}
\frametitle{Rule 6}


\end{frame}
%------------------------------------------------
\begin{frame}
\frametitle{Rule 7}


\end{frame}
%------------------------------------------------
\begin{frame}
\frametitle{Rule 8}


\end{frame}
%------------------------------------------------
\begin{frame}
\frametitle{Rule 9}


\end{frame}
%------------------------------------------------
\begin{frame}
\frametitle{Rule 10}


\end{frame}
%------------------------------------------------
\section{Conclusion}
%------------------------------------------------

\begin{frame}
\frametitle{Conclusions}

My advice is: learn the tools of reproducibility (Box 1) as quickly as possible and use them in every project.
\end{frame}

%------------------------------------------------

\begin{frame}
\frametitle{References}
\fontsize{6}{7.2}\selectfont
\bibliographystyle{apalike}
\bibliography{bib2}
\end{frame}
%------------------------------------------------

\begin{frame}
\frametitle{Acknowledgements}
\begin{itemize}
\item[--] \Large Mike Deery \small{ Manager of the Cambridge Center for Proteomics}
\item[--] \Large Laurent Gatto \small{Computational Proteomics Unit}
\vspace{20px}
\item[--] \Large Andrea Genre \small{Universita' di Torino}
\end{itemize}
\end{frame}



%----------------------------------------------------------------------------------------

\end{document}
